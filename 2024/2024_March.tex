\begin{center}
\section*{\creationmonth}
\end{center}

%\def\day{\textit{\monthdayyeardate\today}}
%\def\weekday{\textit{Tuesday}}
\subsection*{March 25, 2024}
Following the idea proposed in $2312.08228$. 
\subsubsection*{U(1) Model Application}
\begin{itemize}
    \item Conflict between the U(1) model plaquette picture and the link picture.
\end{itemize}

\textbf{Problem:} 
The Action in the U(1) model is a sum of cosine of the angle of the plaquettes, but in the link picture  it is the cosine of the sum of the angles of the links.
%Action Equation
\begin{equation}
    S_{\text{plaquette}}^{2D} = \beta(L^2- \sum_{\text{j}=1}^{L}\sum_{\text{i}=1}^{L} \cos(\theta_{\text{i,j, \,plaquette}}))
\end{equation}
%Link Equation
\begin{equation}
    S_{\text{link}}^{2D} = \beta(L^2- \sum_{\text{j}=1}^{L}\sum_{\text{i}=1}^{L}\cos(\theta_{\text{i,j, \,link}}^{\mu=0}+\theta_{\text{i+1,j, \,link}}^{\mu=1}-\theta_{\text{i,j+1, \,link}}^{\mu=0}-\theta_{\text{i,j, \,link}}^{\mu=1}))
\end{equation}
Can we show that the plaquette picture is equivalent to the link picture in the Haar measure when we compute the partition function?

\textbf{Solution:}
The Haar measure is invariant under the transformation of the link variables. If we choose a gauge where all link variables pointing along the time direction are rotated to the unit element which means \textbf{[Temporal Gauge]} $$\theta^{\mu=0}_{\text{i,j,\, link}} = 0 \,\,\,\forall\,\,\text{i, j}\in[1, L]$$
Then the link action becomes
\begin{equation}
    S_{\text{link}}^{2D} = \beta(L^2- \sum_{\text{j}=1}^{L}\sum_{\text{i}=1}^{L}\cos(\theta_{\text{i,j, \,link}}^{\mu=1}-\theta_{\text{i+1,j, \,link}}^{\mu=1}))
\end{equation}
Now we can see that the link action is equivalent to the plaquette action in the Haar measure.